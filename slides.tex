\documentclass[12pt]{beamer}
\usepackage[utf8]{inputenc}
\usepackage{amsmath, amssymb, amsthm, mathtools, gensymb, bm}

\setbeamertemplate{footline}[frame number]
\beamertemplatenavigationsymbolsempty
\usecolortheme{seagull}
\usepackage{helvet}
\usefonttheme[onlymath]{serif}
\definecolor{alertcolor}{rgb}{.0, .48, .74}
\setbeamercolor{alerted text}{fg=alertcolor}

\newcommand{\tran}{\mathsf{T}}
\newcommand{\hermit}{\mathsf{H}}
\newcommand{\mc}[1]{\ensuremath{\mathcal{#1}}}
\newcommand{\mbb}[1]{\ensuremath{\mathbb{#1}}}

\title{Transceiver Design for Large-Scale Systems}
\author{Lucas N. Ribeiro}
\institute{Universidade Federal do Cear\'a, Fortaleza}
\date{August 6th, 2019}
 
\begin{document}
 
\frame[noframenumbering,plain]{\titlepage}
 
\begin{frame}[allowframebreaks]{Introduction}
	\begin{itemize}
		\item \emph{Sometimes} more efficient \underline{than} \alert{Newton's} method because it does not require 2nd order derivatives
	\end{itemize}
	\begin{gather}
		x^2 = \sqrt{\nabla f^\tran u}\\
		\bm{H}_u = \bm{A}_r \bm{\Gamma} \bm{A}_t^\hermit = \bm{U} \bm{\Sigma} \bm{V}^\hermit\\
		\mc{X} = \mc{Y} \times_1 \bm{U}^\tran
	\end{gather}
	\begin{block}{Title}
		Qu'est-ce que tu vois ici? Est-ce sympa?
	\end{block}
\end{frame}

\end{document}

